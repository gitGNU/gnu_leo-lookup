\documentclass{article}

\usepackage[T1]{fontenc}
\usepackage[utf8]{inputenc}

\title{Documentation of Options for LEO}
\author{Markus Zeindl}

\begin{document}

\maketitle
Following options of LEO have been discovered:
\begin{itemize}
\item{search - Keyword for search}
\item{relink - Controls wheter solutions are links or not}
\item{searchLoc - Direction of search}
\item{Page - Languages}
\end{itemize}

\section{The ``search'' option}

The value of the search option is the word, for what you are looking for.

\section{The ``relink'' option}

This option controls wheter words in the results should be weblinks. The target of the weblinks is the result page of the word on which you have clicked, before.\\
The value of the relink option can be ``on'' or ``off'' without double-quotes.

\section{The ``searchLoc'' option}

The option ``searchLog'' controls in which direction you are looking for. For instance, if you want to know the German meaning of the word ``image'', searchLoc should be zero or -1, standing for English to German.

Possible values:
\begin{table}
  \centering
  \begin{tabular}{|l|l|}

    \textbf{Value} & \textbf{Meaning}\\
   -1     & English to German\\
    0     & Automatic\\
    1     & German to English\\
  \end{tabular}
  \caption{Values of searchLoc}
\end{table}

\section{Languages}

The selection of languages to look for is done by accessing different pages. 
The general URL http://dict.leo.org/ is equal to all languages. The name of the
page the user needs to access is determinated according to this scheme:
The name consists of two parts. The first part is the short of the foreign language and the second is the short of the german language. ``esde'', for example, would lead to a search form between spanish and german.
\end{document}